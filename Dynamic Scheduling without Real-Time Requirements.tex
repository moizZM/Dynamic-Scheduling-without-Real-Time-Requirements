\documentclass[conference]{IEEEtran}
\IEEEoverridecommandlockouts
% The preceding line is only needed to identify funding in the first footnote. If that is unneeded, please comment it out.
\usepackage{cite}
\usepackage{amsmath,amssymb,amsfonts}
\usepackage{algorithmic}
\usepackage{graphicx}
\usepackage{textcomp}
\usepackage{xcolor}
\def\BibTeX{{\rm B\kern-.05em{\sc i\kern-.025em b}\kern-.08em
    T\kern-.1667em\lower.7ex\hbox{E}\kern-.125emX}}
\begin{document}

\title{ Dynamic Scheduling without Real-Time Requirements\\

}

\author{\IEEEauthorblockN{ Moiz Zaheer Malik}
\IEEEauthorblockA{\textit{ B.Eng. Electronic Engineering} \\
\textit{Hochschule Hamm-Lippstadt}\\
Lippstadt, Germany \\
moiz-zaheer.malik@stud.hshl.de}

}

\maketitle

\begin{abstract}
Dynamic scheduling plays an important role in improving system performance in situations where real-time requirements are absent. Without the pressure of strict deadlines, schedulers can be used more effectively to achieve secondary goals, such as reducing power consumption. This is especially relevant in High-Level Synthesis (HLS), which involves three main stages  one of which is scheduling during hardware design. While the initial focus of HLS was on ensuring functional correctness, the need for energy efficiency has grown significantly over time. By dynamically adjusting tasks and deciding which parts of the hardware to use based on the nature of the workload, HLS tools become valuable in saving power without slowing down performance. This highlights the importance of dynamic scheduling in systems that do not rely on real-time constraints.
 In this paper, we explore the role of dynamic scheduling in HLS systems without real-time constraints, focusing on how it can be leveraged to enhance energy efficiency while maintaining performance. We analyze scheduling strategies and demonstrate their impact on power consumption in non-real-time scenarios. 
\end{abstract}

\begin{IEEEkeywords}
component, formatting, style, styling, insert
\end{IEEEkeywords}

\section{Introduction}
 

\section{Ease of Use}

\subsection{}


\section{}


\subsection{}\label{AA}

\subsection{}
\begin{itemize}

\end{itemize}

\subsection{}

\begin{equation}

\end{equation}


\subsection{}




\subsection{}
\begin{itemize}

\end{itemize}


\subsection{}


\subsection{}

\subsection{}


\section*{Acknowledgment}


\section*{}


\begin{thebibliography}{00}
\bibitem{b1} G. Eason, B. Noble, and I. N. Sneddon, ``On certain integrals of Lipschitz-Hankel type involving products of Bessel functions,'' Phil. Trans. Roy. Soc. London, vol. A247, pp. 529--551, April 1955.

\end{thebibliography}
\vspace{12pt}
\color{red}
IEEE conference templates contain guidance text for composing and formatting conference papers. Please ensure that all template text is removed from your conference paper prior to submission to the conference. Failure to remove the template text from your paper may result in your paper not being published.

\end{document}
